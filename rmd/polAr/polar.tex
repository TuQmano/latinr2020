\documentclass[letterpaper]{report}
%\usepackage[utf8]{inputenc}
\usepackage[T1]{fontenc}
\usepackage{RJournal}
\usepackage{amsmath,amssymb,array}
\usepackage{booktabs}

%% load any required packages here

\usepackage[spanish]{babel}
\usepackage{graphicx}

\hypersetup{pdftitle={\{polAr\} Política Argentina Usando R},
            pdfkeywords={r package; elecciones; política}}



\hypersetup{pdfauthor=Anónimo}


%\usepackage[hidelinks]{hyperref}

\urlstyle{same}  % don't use monospace font for urls
\usepackage{color}
\usepackage{fancyvrb}
\newcommand{\VerbBar}{|}
\newcommand{\VERB}{\Verb[commandchars=\\\{\}]}
\DefineVerbatimEnvironment{Highlighting}{Verbatim}{commandchars=\\\{\}} 
% Add ',fontsize=\small' for more characters per line
\usepackage{framed}
\definecolor{shadecolor}{RGB}{248,248,248}
\newenvironment{Shaded}{\begin{snugshade}}{\end{snugshade}}
\newcommand{\AlertTok}[1]{\textcolor[rgb]{0.94,0.16,0.16}{#1}}
\newcommand{\AnnotationTok}[1]{\textcolor[rgb]{0.56,0.35,0.01}{\textbf{\textit{#1}}}}
\newcommand{\AttributeTok}[1]{\textcolor[rgb]{0.77,0.63,0.00}{#1}}
\newcommand{\BaseNTok}[1]{\textcolor[rgb]{0.00,0.00,0.81}{#1}}
\newcommand{\BuiltInTok}[1]{#1}
\newcommand{\CharTok}[1]{\textcolor[rgb]{0.31,0.60,0.02}{#1}}
\newcommand{\CommentTok}[1]{\textcolor[rgb]{0.56,0.35,0.01}{\textit{#1}}}
\newcommand{\CommentVarTok}[1]{\textcolor[rgb]{0.56,0.35,0.01}{\textbf{\textit{#1}}}}
\newcommand{\ConstantTok}[1]{\textcolor[rgb]{0.00,0.00,0.00}{#1}}
\newcommand{\ControlFlowTok}[1]{\textcolor[rgb]{0.13,0.29,0.53}{\textbf{#1}}}
\newcommand{\DataTypeTok}[1]{\textcolor[rgb]{0.13,0.29,0.53}{#1}}
\newcommand{\DecValTok}[1]{\textcolor[rgb]{0.00,0.00,0.81}{#1}}
\newcommand{\DocumentationTok}[1]{\textcolor[rgb]{0.56,0.35,0.01}{\textbf{\textit{#1}}}}
\newcommand{\ErrorTok}[1]{\textcolor[rgb]{0.64,0.00,0.00}{\textbf{#1}}}
\newcommand{\ExtensionTok}[1]{#1}
\newcommand{\FloatTok}[1]{\textcolor[rgb]{0.00,0.00,0.81}{#1}}
\newcommand{\FunctionTok}[1]{\textcolor[rgb]{0.00,0.00,0.00}{#1}}
\newcommand{\ImportTok}[1]{#1}
\newcommand{\InformationTok}[1]{\textcolor[rgb]{0.56,0.35,0.01}{\textbf{\textit{#1}}}}
\newcommand{\KeywordTok}[1]{\textcolor[rgb]{0.13,0.29,0.53}{\textbf{#1}}}
\newcommand{\NormalTok}[1]{#1}
\newcommand{\OperatorTok}[1]{\textcolor[rgb]{0.81,0.36,0.00}{\textbf{#1}}}
\newcommand{\OtherTok}[1]{\textcolor[rgb]{0.56,0.35,0.01}{#1}}
\newcommand{\PreprocessorTok}[1]{\textcolor[rgb]{0.56,0.35,0.01}{\textit{#1}}}
\newcommand{\RegionMarkerTok}[1]{#1}
\newcommand{\SpecialCharTok}[1]{\textcolor[rgb]{0.00,0.00,0.00}{#1}}
\newcommand{\SpecialStringTok}[1]{\textcolor[rgb]{0.31,0.60,0.02}{#1}}
\newcommand{\StringTok}[1]{\textcolor[rgb]{0.31,0.60,0.02}{#1}}
\newcommand{\VariableTok}[1]{\textcolor[rgb]{0.00,0.00,0.00}{#1}}
\newcommand{\VerbatimStringTok}[1]{\textcolor[rgb]{0.31,0.60,0.02}{#1}}
\newcommand{\WarningTok}[1]{\textcolor[rgb]{0.56,0.35,0.01}{\textbf{\textit{#1}}}}

\providecommand{\keywords}[1]{\noindent\textbf{Palabras clave:} #1}
\providecommand{\tightlist}{%
\setlength{\itemsep}{0pt}\setlength{\parskip}{0pt}}


\begin{document}

%% do not edit, for illustration only
\sectionhead{\{polAr\} Política Argentina Usando R}
\year{2020}

\begin{article}

\title{\{polAr\} Política Argentina Usando R}


\author{Anónimo}

\maketitle


\keywords{ r package  -  elecciones  -  política }

\hypertarget{abstract}{%
\section{Abstract}\label{abstract}}

\CRANpkg{polAr} es un paquete pensado para para facilitar el flujo de
trabajo y el acceso a datos político-electorales de Argentina. Entre
otros, está inspirado en los paquetes \CRANpkg{eph} (Kozlowski et al.
2019) -que facilita el acceso a datos de la \emph{Encuesta Permanente de
Hogares} del \emph{Instituo Nacional de Estadísticas y Censos} de
Argentina- y \CRANpkg{esaps} (Schmidt 2018) -que proveé métodos para el
cómputo de indicadores de sistemas de partidos y electorales.

\hypertarget{datos}{%
\subsection{1.Datos}\label{datos}}

Aunque \CRANpkg{polAr} no es un paquete de datos, una de sus principales
funciones es facilitar el acceso a los mismos haciendo llamadas a un
repositorio independiente. El código y diseño de \CRANpkg{eph} fueron
centrales para desarrollar esta parte. El primer paso consistió en el
pre procesamiento de las bases de datos originales\footnote{La fuente
  original de datos para resultados de elecciones nacionales (2003-2017)
  provienen del \emph{Atlas Electoral} de Tow (2020). Los datos de las
  elecciones de 2019 tienen una estructura diferente de las de años
  anteriores y fueron reconstruidos de unos paquetes especificos
  desarrollados por Moracho (2020) para cada turno: \emph{P.A.S.O.} y
  \emph{Generales}. Más detalles disponibles en el repositorio de datos:
  https://github.com/electorArg/PolAr\_Data/.} y el diseño de una
estructura de archivos que nos permitiera hacer esas llamadas.
\texttt{show\_available\_elections()} es una función que devuelve una
tabla que funciona como índice de elecciones disponibles. El mismo está
basado en el nombre de archivos de esa estructura, y son a la vez los
parámetros necesarios para descargar una elección:

\begin{Shaded}
\begin{Highlighting}[]
\KeywordTok{library}\NormalTok{(polAr)}

\NormalTok{tucuman2019Dip <-}\StringTok{ }\KeywordTok{get_election_data}\NormalTok{(}\DataTypeTok{district =} \StringTok{"tucuman"}\NormalTok{,}
                  \DataTypeTok{category =} \StringTok{"dip"}\NormalTok{,}
                  \DataTypeTok{round =} \StringTok{"paso"}\NormalTok{,}
                  \DataTypeTok{year =} \DecValTok{2019}\NormalTok{)}
\end{Highlighting}
\end{Shaded}

La idea general es disponibilizar la información lo más desagregada y
limpia posible para usar del modo más conveniente por cada usuarie. Es
por ello que se agregan opciones para obtener la \emph{data}
\emph{cruda} (usando el parámetro \texttt{raw\ =\ TRUE}) o descargarla
en formato \emph{ancho} (con parámetro \texttt{long\ =\ FALSE}). Además
de funciones auxiliares como \texttt{make\_long()} para pasar a formato
\emph{tidy} un \texttt{data.frame} \emph{ancho} o \texttt{get\_names()}
que permite agregar los nombres a los \emph{id} de listas presentes en
la \emph{data.}

\hypertarget{indicadores}{%
\subsection{2.Indicadores}\label{indicadores}}

Otra alternativa, siguiendo las ideas propuestas en \CRANpkg{esaps}, es
poder calcular indicadores a partir de la información obtenida. En esta
versión es posible computar \emph{Competitividad} o el \emph{Número
Efectivo de Partidos}, de este modo:

\begin{Shaded}
\begin{Highlighting}[]
\KeywordTok{compute_nep}\NormalTok{(tucuman2019Dip)}
\end{Highlighting}
\end{Shaded}

\begin{verbatim}
## # A tibble: 2 x 3
##   codprov value index           
##   <chr>   <dbl> <chr>           
## 1 23       2.17 Golosov         
## 2 23       2.80 Laakso-Taagepera
\end{verbatim}

\hypertarget{visualizaciuxf3n}{%
\subsection{3.Visualización}\label{visualizaciuxf3n}}

Por último, con esta misma \emph{data} podemos hacer uso de funciones
que permiten visualizar rápidamente los datos como tablas
(\texttt{tabulate\_results}), mapas (\texttt{map\_results}) o gráficos
(\texttt{plot\_results}):

\begin{Shaded}
\begin{Highlighting}[]
\KeywordTok{plot_results}\NormalTok{(tucuman2019Dip)}
\end{Highlighting}
\end{Shaded}

\includegraphics{polar_files/figure-latex/plot-1.pdf}

\hypertarget{referencias}{%
\section*{Referencias}\label{referencias}}
\addcontentsline{toc}{section}{Referencias}

\hypertarget{refs}{}
\leavevmode\hypertarget{ref-eph}{}%
Kozlowski, Diego, Pablo Tiscornia, Guido Weksler, Natsumi Shokida, and
German Rosati. 2019. \emph{Eph: Argentina's Permanent Household Survey
Data and Manipulation Utilities}.
\url{https://CRAN.R-project.org/package=eph}.

\leavevmode\hypertarget{ref-pmoracho}{}%
Moracho, Patricio. 2020. \emph{Elecciones.ar.2019: Elecciones Nacionales
de Argentina 2019}. \url{http://github.com/pmoracho/elecciones.ar.2019}.

\leavevmode\hypertarget{ref-esaps}{}%
Schmidt, Nicolas. 2018. \emph{Esaps: Indicators of Electoral Systems and
Party Systems}. \url{https://CRAN.R-project.org/package=esaps}.

\leavevmode\hypertarget{ref-andytow}{}%
Tow, Andy. 2020. \emph{Atlas Electoral}. \url{https://www.andytow.com/}.




\end{article}
\end{document}

